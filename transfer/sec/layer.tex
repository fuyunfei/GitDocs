\chapter{LAYER DECOMPOSITION}

Digital painting with different layers is an integral feature of digital image editing software, such as Photoshop and Sketchbook. Layers offer an intuitive way to edit the color and geometry of components and localize changes to the desired portion of the final image. Without layers, brush stroke segmentation becomes extremely challenging, since they can overlap and blend with each other.

In general, each layer represents one coat of a painting with single color that is applied with varying opacity throughout the input painting. Wrong layer decomposition may cut one stroke into different layers. It is crucial to preserve the completeness and smoothness of the brush strokes in layer decomposition. To this end, we modify the layer decomposition algorithm in  by involving the coherent lines \cite{kang2007coherent} in our implementation. In the following we first address the layer decomposition algorithm briefly and then discuss our modification. 


\section{Identify Paint Colors}

\begin{figure}[H]
	\centering
	\includegraphics[height=0.2\textheight]{5.png}
	~~~~
	\includegraphics[height=0.2\textheight]{pcloud.png}
	\caption{Geometry of input image pixels in RGB-space}
	\label{point cloud}
\end{figure}

In a brush painting, one region may have been painted multiply time with different paint colors. We assume that the color on each pixel is a linear combination of the paint colors, so all the pixel color in the input painting lies the convex hall of in the RGB space as showed in Figure \ref{point cloud}. And base on such idea, we can represent each pixel color based on painted color: 
\begin{equation*}
p=\Sigma \omega_i c_i
\end{equation*}
$p$ represents the color of the pixel, and $c_i$ represents the $i$-th paint color. 
To compute the paint color, we introduced the convex hull simplifying method of Tan's work \cite{tan2016decomposing}. In which a convex hull of the colors in RGB space should be computed, while every vertice is considered as a paint color. The colors would be tightly wrapped by the convex hull, but normally there would be many vertices more than what we need, since too many vertices would result in too many layres. In Tan's work they provide a simplification method which would output manageable number of layers based on user need and the output layers with clearly differentiated colors,as showed in Figure \ref{convex}.

\begin{figure}[H]
	\centering
	\includegraphics[height=0.18\textheight]{convex.jpg}
	\caption{Convex hull of input image}
	\label{convex}
	\medskip
	\small
	(a) A simple digital painting’s (b) convex hull in RGB-space is complex due to rounding. (c) The result of simplification algorithm 
\end{figure}

\section{Layer Decomposition Scheme}


The “A over B” compositing and blend mode in \cite{porter1984compositing} is described that when the pixel $A$ with color $A_{RGB}$ and translucency $\alpha_{A}$ is placed over the pixel $B$ with color $B_{RGB}$ and translucency $\alpha_{B}$, the observed color is, 
\[  \left ( \frac{A}{B} \right )_{RGB}=\frac{\alpha_{A}A_{RGB}-(1-\alpha_{A})\alpha_{B}B_{RGB}}{\left ( \frac{A}{B} \right )_{\alpha}} \]
where 
\[\left ( \frac{A}{B} \right )_{\alpha} = \alpha_{A}+(1-\alpha_{A})\alpha_{B} \] 
Each pixel’s color is viewed as the convex combination of all layers’ colors. For each pixel, the observed color p can be approximated by the recursive application of the compositing and blend model. We take as input ordered RGB layer colors through computing per-pixel opacity values for each layer. The following ‘polynomial’ regularization term penalizes the difference between the observed color p and the polynomial approximation,

\[E_{polynomial}=\frac{1}{K}\left \|    C_{n}+\sum_{i=1}^{n}  \left ( \left ( C_{i-1}-C_{i} \right ) \prod_{j=i}^{n}(1-\alpha_{j}) \right )-p  \right \|^{2}\]

where $C_{i}$ denotes the i-th layer’s color, is the opacity of $\alpha_{i}$ , the background color $C_{i}$ is opaque, $K=3$ and or 4 depending on the number of channels $(RGB$ or $RGB-\alpha)$. The opacity penalty is expressed as,
\[E_{opaque}=\frac{1}{n}\sum_{i=1}^n-(1-\alpha_{i})^2\]
The default smoothness penalty is expressed as,

\[E_{spatial}=\frac{1}{n}\sum_{i=1}^n( \bigtriangledown  \alpha_{i})^2\]

where $ \bigtriangledown  \alpha_{i}$is the spatial gradient of opacity in the i-th layer. This term penalizes solutions which are not spatially smooth. However, the gradient of opacity is not always aligned with that of intensity, which may result in edges discontinuous.
The users may specify the layer order in advance, as well as the number of layers, n, is given. The opacity for every layer may be solved by minimizing the following combined cost function,
\begin{equation}
E=\omega_{polynomial}E_{polynomial}+\omega_{opaque}E_{opaque}+\omega_{spatial}E_{spatial}
\label{eq:layer1}
\end{equation} 
where $\omega_{polynomial} = 375 ,\omega_{opaque}=1 , \omega_{spatial}=10$. 

\subsection{Modified Layer Decomposition}
\begin{figure}[H]
	\centering
	\includegraphics[width=4cm]{5.png}
	~~~~
	\includegraphics[width=4cm]{5etf.png}
	\caption{Edge Tangent Flow}
	\label{ETF}
\end{figure}

As we can see in Figure \ref{decom:4layers},the layer decomposited , to enhance the smoothness and completeness of strokes, the coherent line drawing technique in \cite{kang2007coherent} is introduced to Eq \ref{eq:layer1} ,which is a flow-guided anisotropic filtering framework. Figure \ref{ETF} shows the edge tangent flow $($ETF$)$ field of a Rosemaling painting. First, the ETF field is involved in $E_{spatial}$.\newline
The ETF field is defined as,
\begin{equation}
 t^{new(x)}=\frac{1}{k}\sum_{y\in\Omega(x)} \varphi(x,y)t^{current}(y)\omega_{s}(x,y)\omega_{m}(x,y)\omega_{d}(x,y)
 \label{eq:layer_etf} 
\end{equation}

As showed in Figure \ref{Coherent line details }, $t(x)$ denotes the normalized tangent vector at pixel $x$, $\Omega(x)$ denotes the neighborhood of the pixel $x$, and $k$ is the term of vector normalization. The spatial weight function $\omega_{s}$ employs the radially-symmetric box filter with some radius. The magnitude weight function $\omega_{m}$ is monotonically increasing, indicating that the bigger weights are given to the neighboring pixels y whose gradient magnitudes are higher than that of the central pixel x. This ensures the preservation of the dominant edge directions. The direction weight function, $\omega_{d}$, may enhance alignment of vectors, e.g. $t(x)\cdot t(y)>0$ , while suppressing swirling flows. In addition, the sign function $\varphi(x,y)$  is employed to prevent the swirling artifact as well.

\begin{figure}[H]
	\centering
	\includegraphics[width=10cm]{ETF_explain.png}
	\caption{Coherent line details }
	\label{Coherent line details }
	\medskip
	\small
	(a) Input (b) ETF (c) Kernel at x (d) Kernel enlarged (e) Gaussian components for DoG
\end{figure}

\begin{figure}[H]
	\centering
	\includegraphics[width=5cm]{com_layer1.png}
	\includegraphics[width=5cm]{com_layer2.png}
	\caption{Comparison of layer decomposition  }
	\label{com eflow}
	\medskip
	\small
	Comparison of layer decomposition before and after modification  at the 2nd layers. left and right show the results by using $E_{flow}$ instead of $E_{spatial}$ in Eq \ref{eq:layer1};
\end{figure}

Involving ETF filed of Eq \ref{eq:layer_etf} in $E_{spatial}$, the smoothness penalty is rewritten as,

\begin{equation} 
E_{flow}=\frac{1}{n} \sum_{i=1}^{n} \left \| t^{new} \right \| \left ( \bigtriangledown_{\theta}\alpha_{i} \right )^2 
\end{equation} 

where $\theta$ denotes the direction of $t^{new}$, and $ \bigtriangledown_{\theta}\alpha_{i} $ is the gradient of opacity in the direction of $t^{new}$. Moreover, we weight this penalty by the norm of $t^{new}$. Applying the updated $E_{flow}$ to the layer decomposition of Eq \ref{eq:layer1} instead of $E_{spatial}$, the strokes become complete and smooth, which can be noted in Figure \ref{com eflow}.

\begin{figure}[H]
	\centering
	\includegraphics[width=5cm]{5etf.png}
	\includegraphics[width=5cm]{edge.png}
	\caption{Edge Tangent Flow field and coherent lines of a Rosemaling painting. It contains lots of C and S strokes.}
\end{figure}


Second, the coherent lines as the constraint of brush stroke edges are involved in layer decomposition of Eq \ref{eq:layer1}. Herein, the coherent lines can be computed as follows.
\begin{figure}[H]
	\centering
	\includegraphics[width=5cm]{com_layer3.png}
	\includegraphics[width=5cm]{com_layer4.png}
	\caption{Comparison of layer decomposition before and after using $E_{egde}$  in Eq \ref{eq:layer1}; }
	\label{com:edge}
\end{figure}
Given a ETF field $t(x)$, the flow-guided anisotropic Difference of Gaussian (DoG) filter is employed, in which the kernel shape is defined by the local flow encoded in ETF field. Note that $t(x)$ represents the local edge direction. It is most likely to make the highest contrast in the perpendicular direction, that is, the gradient direction. When moving along the edge flow, the DoG filter is applied in the gradient direction. As a result, we can ‘exaggerate’ the filter output along genuine edges, while ‘attenuating’ the output from spurious edges. This not only enhances the coherence of the edges, but also suppresses noises. Iteratively applying this flow-based DoG filter results in a binary output which reaches a satisfactory level of line connectivity and illustration quality. The coherent lines can be regarded as the edges of brush strokes.

\begin{figure}[H]
	\centering
	\begin{subfigure}[b]{0.3\textwidth}
		\centering
		\includegraphics[width=\textwidth]{alpha.png}
		\caption{alpha map of layer1}

	\end{subfigure}
	~  
	\begin{subfigure}[b]{0.3\textwidth}
		\centering
		\includegraphics[width=\textwidth]{intensity.png}
		\caption{intensity map of layer1}
 
	\end{subfigure}
\end{figure}

\begin{figure}
	\centering
	\begin{subfigure}[b]{0.7\textwidth}
		\includegraphics[clip,width=\textwidth]{h1.png}
		\caption{histogram of intensity map of layer1}
	\end{subfigure}
	
	\begin{subfigure}[b]{0.7\textwidth}
		\includegraphics[clip,width=\textwidth]{h2.png}
		\caption{histogram of transparency map of layer1}
	\end{subfigure}
	\caption{Comparison between transparency and intensity of layer1}
	\label{histo}
\end{figure}



 
To preserve the stroke edges, we assume that the opacity along the coherent lines is consistent, i.e. $min \int_{l}\Arrowvert \bigtriangledown\alpha \Arrowvert^2 dx $, where l denotes the collection of coherent lines. Hence, the constraint term is defined by applying Laplacian operator to the opacity along the coherent lines,

\begin{equation} 
E_{edge}=\Arrowvert LY \Arrowvert^2 
\end{equation} 

where all the opacity $\alpha_i$ are stacked in the vector $Y$, and $L$ denotes the connection matrix. The eight-connected neighboring rule is utilized to construct the connection matrix $L$, that is, if two adjacent pixels, i and j, stay on the same coherent line, the item of $L(i,j)$ is set to -1 ; otherwise 0. Figure \ref{com:edge}a and \ref{com:edge}b shows that the edges of strokes become visible and complete after involving $E_{edge}$ into Eq \ref{eq:layer1}.
Accordingly, the layer decomposition of Eq \ref{eq:layer1} is rewritten as,

\begin{equation}
E=\omega_{polynomial}E_{polynomial}+\omega_{opaque}E_{opaque}+\omega_{flow}E_{flow}+\omega_{edge}E_{edge}
\label{eq:layer_sum}
\end{equation} 

where $\omega_{flow}=100, \omega_{edge}=20 $ for all our examples. Figure 8 shows an evaluation of the effect of changing weights for our one example in which the defaults do not produce
the best output.
 
\begin{figure}[!htb]
 
	\begin{minipage}[t]{0.5\linewidth}
		\centering
		\includegraphics[width=2.2in]{modi_wflow.png}
		\caption{fig1}
		\label{fig:side:a}
	\end{minipage}%
	\begin{minipage}[t]{0.5\linewidth}
		\centering
		\includegraphics[width=2.2in]{modi_wflow.png}
		\caption{fig2}
		\label{fig:side:b}
	\end{minipage}
 
%%\centering
%%\begin{minipage}[c]{\textwidth}
%%	\centering
%%	\includegraphics[width=2.0in]{modi_wflow.png}
%%	\caption{Caption for image}
%%	\label{fig:sample_figure}
%%\end{minipage}
%%\begin{minipage}[c]{\textwidth}
%%	\centering
%%	\includegraphics[width=2.0in]{modi_wflow.png}
%%	\caption{Caption for image}
%%	\label{fig:sample_figure}
%%\end{minipage}
%%\begin{minipage}[c]{\textwidth}
%%	\centering
%%	\includegraphics[width=2.0in]{modi_wflow.png}
%%	\caption{Caption for image}
%%	\label{fig:sample_figure}
%%\end{minipage}	
%	\centering
%
%	\begin{subfigure}[b]{0.7\textwidth}
%		\includegraphics[clip,width=\textwidth]{modi_wflow.pdf}
%		\caption{histogram of intensity map of layer1}
%	\end{subfigure}
%%
%	\begin{subfigure}[b]{0.7\textwidth}
%		\includegraphics[clip,width=\textwidth]{modi_wopaque.pdf}
%		\caption{histogram of transparency map of layer1}
%	\end{subfigure}
%	
%	\begin{subfigure}[b]{0.7\textwidth}
%		\includegraphics[clip,width=\textwidth]{modi_wedge.pdf}
%		\caption{histogram of transparency map of layer1}
%	\end{subfigure}
%	
%	\caption{Comparison between transparency and intensity of layer1}
%	\label{modi_para}
\end{figure}













For comparison, we perform the schemes of Eq \ref{eq:layer1} and Eq \ref{eq:layer_sum} separately on the same set of brush paintings and compare the root-mean-square-error (RMSE) of the opacity of the coherent lines on each layer shown in Table 1. The RMSE by Eq \ref{eq:layer_sum} is noticeably less than that by Eq \ref{eq:layer1}. This means that the coherent lines have been embedded into the opacity of each layer. The weights are empirically determined in terms of the opacity RMSE of coherent lines.

\begin{figure}
	\centering
	\includegraphics[width=17cm]{rmse.pdf}

\end{figure}

















