\chapter{BAS-RELIEF GENERATION BASED ON BRUSH STROKE }
Since the brush strokes can be extracted individually, it is natural to independently generate the individual depth maps and then merge the depth maps together to form the desired bas-relief, see Figure \ref{decom:overview}.It is noteworthy that we use a depth map to represent a bas-relief model,which is commonly used for bas-relief generation methods.   In this Chapter, we demonstrate how to generate bas-relief from brush strokes, and how it can be modified. We also compare our algorithm against the two most notable 2D image based bas-relief generation algorithm. \\
In our implementation, we employ the orthogonal SFS \cite{prados2004unifying} on the segmented brush strokes. The brightness equation used in SFS is expressed as,

\begin{equation*}
I(x)=\frac{1}{\sqrt{1+\lvert \bigtriangledown h \rvert ^2}}
\end{equation*}
$I(x)$ is the intensity at pixel x, and it can be noted that for higher intensity $I$, change of depth $h$ is smaller.Some brush strokes are painted by colors with high intensity. As a result, if the shape from shading algorithm is performed on intensity, the resulting stroke models will become flat and lack of hierarchy. The opacity of image is independent of the color intensity (see Figure \ref{histo}). Each stroke has an appropriate distribution of opacity, which is in favor of a layered look, so we reformulate the equation : 
\begin{equation*}
\alpha(x)=\frac{1}{\sqrt{1+\lvert \bigtriangledown h \rvert ^2}}
\end{equation*}
$\alpha(x)$ is the opacity value of pixel $x$ on a brush stroke. 
To make the bas-relief more inflated,we rewrite the as,

\begin{equation}
\lVert \bigtriangledown h \rVert = \sqrt{\frac{1}{\lVert \alpha(x) \rVert ^2}-1+ \Delta}
\end{equation}
where $\Delta$ is a positive displacement. This modification may make the surface inflated. By solving this equation, we can generated a depth map for each brush stroke. As show in Figure \ref{strokerelief}, we input the opacity map of a decomposed layer, and from that layer we extract three brush strokes shown in different color,and by apply our algorithm we can generate bas-reliefs(depth maps) from each one of the brush strokes on this layers, and by generate bas-reliefs from all brush strokes, and merge them together, we automatically generate a bas-relief from the input Chinese painting. \\
To differentiate the bas-relief generated form a single stroke and the bas-relief generated from the whole painting, we use the term "stroke-relief" to indicate a bas-relief(depth map) generated from a single stroke. And just like a Chinese painting can be regarded as a union of brush strokes \cite{xu2006animating} ,in our approach, a bas-relief can be regarded as a union of stroke-relief.
\begin{figure}[H]
	\centering
	\includegraphics[width=15cm]{strokerelief.png}
	\caption{Bas-relief generation from brush strokes}
	\label{strokerelief}
\end{figure}


\section{Comparisons}\label{compare}

In this section, we compares our bas-relief generation method against another more recent and notable 2D image based bas-relief generation method \cite{zeng2014region}. \\
Zeng et al.'s work \cite{zeng2014region} present region-based algorithm for bas-relief generation.At first,their method extracts 2D regions from input image and determines the layer of the regions. Then, they reconstruct relief from each layer. And they demonstrate this algorithm works well for a range of input photos, including human faces, flowers and animals.\\
In their method, to extract regions from the image, the feature lines are detected at first. Next, seed points are derived from these lines, and regions are found from them using a region growing process. The region extraction in their method is quite similar with our brush stroke extraction process. For Chinese painting, a brush stroke can be considered as a region.So we compare our method against Zeng et al.'s method \cite{zeng2014region} based on three factors mentioned in Section \ref{2dimagebased} : depth information, silhouettes and edges,fine details. \\
\textbf{Depth information:} In the Zeng et al.'s work \cite{zeng2014region} , regions determine the depth information, so regions have to be faithfully extracted. The first step of extract regions is detect region outlines by line detection on intensity map. Naturally, the line detection fail in following scenarios: \\
(1) The brush stroke with the intensity very close to the background; \\
(2) Two adjacent brush stroke painted by different colors with the similar intensity;\\
So, their algorithm would fail at the first step.\\
In our algorithm, the above mentioned problems is avoided by layer decomposition and stroke extraction , see Section \ref{extract}. By extracting brush stroke faithfully, we follow the manner of Chinese painting, generate bas-relief from all brush strokes. \\
\textbf{Silhouettes and Edges:} As mention above, the region-based algorithm fails to find the outlines of brush strokes, namely, it can not separate brush strokes from the background or generate a reasonable contrast between the foreground objects and the background as a starting establishment for interpreting heights.\\
\textbf{Fine Details:} Region-based algorithm performs shape from shading on the intensity of the image.As we mentioned,for brush strokes painted by bright color, the generated bas-relief would be flat and features can not be well preserved. In our algorithm, we use opacity to generate the bas-relief, which maintains the details of brush strokes better.  \\

 