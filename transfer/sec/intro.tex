\chapter{INTRODUCTION}

\section{Background}

Relief is a kind of sculpture in which 3D sculpture are carved into a relatively flat surface. In essence, it creates a bridge between a full 3D sculpture and a 2D painting. On this spectrum,high relief is closest to full 3D, whereas flatter artworks are described as bas-relief. Among all the sculpture forms, bas-relief is the closest to 2D paintings\cite{kerber2009feature} \cite{barron2012color}.\\ \\
Bas-relief sculpting has been practiced for thousands of years.Since antiquity, artisans from many ancient cultures(including Greek, Persian, Egyptian, Mayan, and Indian art) have created bas-reliefs.
Today bas-reliefs are commonly found in a variety of media, commemorative medals, coins, souvenirs, 2.5D animation,and artistic sculptures for blind people. However, crafting bas-reliefs is a laborious, challenging and time consuming process. And with the commonly and cheaply available 3D printing facilities, there is a growing trend in the need of bas-relief art products.There are many ways to reach the same goals more easily with the help of computers. In the past two decades, a sizable amount of research has gone into developing bas-relief generation methods\cite{benzaid2017analysis}. It is noteworthy that a 3D sculpture model can alway be represented by a 3D model, and a bas-relief are generally considered a depth map in previous studies.In this research, we use depth map to represent bas-relief surface as well.\\ \\ 
Bas-relief is regarded as an art form part way between 3D sculpture and 2D painting \cite{benzaid2017analysis}\cite{barron2012color}\cite{weyrich2007digital}\cite{kerber2009feature}\cite{kerber2012computer}.In general, the existing bas-relief generation methods can also be classified in two different categories with respect to their input\cite{benzaid2017analysis}:
\begin{itemize}
 \item 3D model based : using a 3D model (sculpture) as input 
 \item 2D image based : using a 2D image as input \\ \\
\end{itemize} 
\textbf{3D model based:}\\ How to generate a bas-relief from a 3D sculpture? The 3D model based bas-relief generation methods focus on such a problem,in which a 3D sculpture are considered as a 3D digital model.   The generation of bas-relief from 3D model was first studied in the pioneering work of \cite{cignoni1997computer}, then various existing 3D model based methods have demonstrate how to compress 3D model into bas-relief \cite{weyrich2007digital}\cite{kerber2009feature}\cite{song2007automatic}\cite{sun2009bas} . However,this approach requires a 3D model as a starting point. \\ \\
\textbf{2D image based:}\\On the other hand, how to generate a bas-relief from a 2D painting? \\ There have been some bas-relief generation approaches available based on 2D image \cite{zeng2014region}\cite{wu2013making} and \cite{alexa2010reliefs}.  These approaches almost follow the “bas-relief ambiguity”\cite{belhumeur1999bas}, that is, roughly speaking, from a frontal view the sculpture looks like a full 3D object while a side-view reveals the disproportional depth. 
A 2D painting can be considered as a image,however,these image based methods are focusing on general photograph from real scene,assuming illumination and reflectance are known, and the image is formed from lighting and shading, which obviously unsuited for 2D paintings.And the generated bas-relief is a single depth map or mesh, which makes it hard for user to edit. 
Some research focus on bas-relief generation from 2D image of line drawing \cite{kolomenkin2011reconstruction}\cite{varley2002estimating}\cite{malik1987interpreting}\cite{sykora2014ink}.However,line drawing based methods do not consider how to generate bas-reliefs with surface details: their approaches are limited to using information contained in a line drawing, which are not suited for paintings which contain information such as color,texture and stroke shape,etc. In general,the generated bas-relief are represented by a depth map or mesh.  \\ \\  
As mentioned above, bas-relief is a art form part way between 2D painting and a full 3D sculpture\cite{benzaid2017analysis}\cite{weyrich2007digital}\cite{kerber2009feature}\cite{kerber2012computer}\cite{zeng2014region} .And among all the sculpture forms, bas-relief is the closest to 2D paintings,as claimed by \cite{kerber2009feature} \cite{barron2012color}, while how to generate bas-relief from artistic paintings remains a problem . \\ \\  
The aim of our research is to provide a method to generate a bas-relief from a Chinese painting. We also argue that because most Chinese paintings are produced with individual brush strokes, generating bas-relief surface from each brush stroke would preserve the original features of the painting.\\
A Chinese painting can be regarded as the union of brush strokes \cite{xu2006animating}. Most Chinese paintings are typically sparse with each brush stroke drawn very purposefully\cite{smith1997art},and each stroke exists on its own as a piece of art\cite{girshick2004simulating}.
Differing from the other bas-relief generation methods, our method will honor this very feature by generation bas-relief surfaces from the brush strokes individually. 
This however demands to conquer several challenges.\\ 
First, each brush stroke covers a region on the canvas and they may overlap each other, some quite heavily in a painting. To make sure the information is retained, every brush stroke has to be faithfully extracted.\\ Second,unlike previous 2D image based methods focusing on photograph, a painting does not obey the rules of lighting and shading, which increase the difficulty to mimic the details on bas-relief surface. \\Third, the generated bas-relief should be further editable allowing the artist to rearrange, tweaking and reshaping the bas-relief surface correspond to each brush strokes. \\  \\
To extract the strokes from a brush painting, we need to identify and segment the overlapped strokes at first. We will then generate the depth map based on every strokes' opacity separately . All the brush strokes' depth map are finally merged together to yield the resulting bas-relief with the feature of original 2D painting preserved.

\section{Motivation}
The motivation of this research comes from three sides.\\First,as mentioned above, bas-relief is regarded as a art form part way between 2D painting and a full 3D sculpture, as claimed by many previous works \cite{benzaid2017analysis}\cite{weyrich2007digital}\cite{kerber2009feature}\cite{kerber2012computer}\cite{zeng2014region}. Research so far has been primarily done based on 3D model,and some other methods based on a single photograph or line drawing as input. However,how to generate bas-relief from artistic paintings remains a problem .\\
Second, due to the various styles of painting, it is difficult to come up with a general bas-relief generation method suitable for all 2D paintings. Some paintings may be too abstract for bas-relief generation.One the other hand, for traditional Western painting style developed in the Renaissance,which emphasizes realism\cite{chu2004real}, previous 2D image based methods may easily handle. Most Chinese paintings are typically sparse with each brush stroke drawn very purposefully\cite{smith1997art}, and such a feature can be found other painting styles,such as rosemailing painting and certain watercolor paintings. Study bas-relief generation from Chinese painting will naturally push forward the research frontier of bas-relief generation from other painting. \\
Third, Bas-relief generation has wide applications in making objects such as commemorative medals, coins, souvenirs, and artistic sculptures for blind people.With the commonly and cheaply available 3D printing facilities, there is a growing trend in the need of bas-relief art products. On the other hand, the style and philosophies of Chinese paintings have influenced other painting styles \cite{chu2004real}, and as input of our bas-relief generation method, the digital image of Chinese paintings are easily accessible on the Internet.    \\
Therefore, designing an efficient method for bas-relief generation from Chinese painting is important both theoretically and practically.


\section{Contribution of Current Work}
In this research, we propose a bas-relief generation method from a Chinese painting. As mentioned above, previous bas-relief generation methods have demonstrate how to generated bas-relief from 3D models,photographs and line drawings. In contrast, this research is the first attempt to generate bas-relief from a artistic painting. \\ Accompanying with this method, we have propose an algorithm for brush stroke extraction. To the best of our knowledge, there is lack of study on extracting brush strokes from paintings.The two most notable brush stroke extraction algorithms are proposed by \cite{li2012rhythmic} and \cite{xu2006animating}. \cite{li2012rhythmic} proposed a brush stroke extraction method by edge detection and clustering-based segmentation, but their method does not support overlap strokes.\cite{xu2006animating} proposed a algorithm for brush stroke extraction using segmentation techniques, however, this method requires professional artists to build a brush stroke library which makes it a challenging and time consuming process. \\ 
The major difference between our brush stroke extraction with the previous works is that our algorithm is based on a reformulated layer decomposition algorithm, which makes it capable of extraction overlapped brush strokes without the prior knowledge of brush strokes.\\
In contrast with previous 2D image based methods, our algorithm is the first attempt to generate bas-relief from opacity of brush strokes. And by generating bas-relief surface for each brush stroke , the combined final bas-relief is further editable allowing the artist to rearrange, tweaking and reshaping.


Our contributions include,
\newline
(1) Extraction of brush strokes. We develop a novel method to extract brush strokes based on palette color analysis and layer decomposition.  Comparing with previous method brush stroke extraction methods, our method is capable of extracting overlapped brush strokes automatically without the prior knowledge of brush strokes. 
\newline
(2) Bas-relief generation of each brush stroke. We develop a novel method which may entirely generated every stroke to a correspond bas-relief surface(a depth map). By doing so, every single bas-relief surface of corresponding brush stroke can be editable, making it a useful tool for sculpture artists.\\
(3) Bas-relief generation from opacity. In contrast with previous 2D image based methods, our method use opacity instead of intensity to generate bas-relief, which can better preserve the feature of input painting. 

Our experiments show that our method can effectively generate digital bas-reliefs for a range of input images, including Chinese paintings and some rosemailing paintings. 
We also demonstrate the utility of the decomposed brush strokes for image recoloring and image object insertion and animation. See section !!! 


\section{Document Structure}
The organization of the document is as follows:

Chapter 1: Introduction. This section provides the background, the motivation and the contribution made in current research.

Chapter 2: Literature Review. This section classifies and reviews related previous works  in a systematic way.

Chapter 3: Overview of proposed approach. This section explains the methodology selected and defines some basic concepts used in the research.

Chapter 4: Layer Decomposition. This section introduces concept "Layer Decomposition" in this research, which is the basis of the proposed algorithm.

Chapter 5: Extraction of Brush strokes. This section describes the proposed algorithm of brush stroke extraction.

Chapter 6: Depth map and '3D strokes'.  This section shows how to generate the individual depth maps of extracted strokes. 

Chapter 7: Results. This section reviews the up-to-date results based on proposed the method.

Chapter 8: Conclusions . This section concludes the progress up-to-date.

Chapter 9: Future work. This section discusses possible future work about high relief and 3D painting.

\newpage