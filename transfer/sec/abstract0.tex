\section*{\centering ABSTRACT}
Bas-relief is an art form part way between 3D sculpture and 2D painting. We present a novel approach for generating a bas-relief from a single Chinese painting.We do not aim to recover exact depth of a painting, which is a tricky computer vision problem, requiring assumptions that are rarely satisfied.Instead, our approach exploits the concept of brush strokes, making each brush stroke possible to generate a correspond bas-relief proxies(depth map of brush strokes),and combines the depth maps of brush strokes together to construct the bas-relief model. To segment brush strokes in 2D paintings, we apply layer decomposition and stroke segmentation by imposing boundary constraints. The resulting brush strokes are sufficient to evoke the impression of input Chinese painting. Currently, our research focus on Chinese paintings.We demonstrate our approach is able to produce convincing bas-reliefs from a variety of Chinese paintings(with human, animal, flower,etc.), and even some other suitable styles include rosemaling paintings. \\
As a secondary application, we show how our brush stroke extraction algorithm could be used for image editing.And our brush stroke extraction algorithm is specifically geared to handling paintings with each brush stroke drawn very purposefully, in addition to Chinese paintings, other suitable styles include rosemailing paintings and Vincent van Gogh's oil paintings.

\newpage


 

