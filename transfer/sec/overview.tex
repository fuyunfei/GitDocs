\chapter{OVERVIEW OF PROPOSED METHOD} \label{overview}

\begin{figure}
\centering
\includegraphics[height=0.8\textheight]{lotusoverview.png}
\caption{ Overview of Proposed Method}
\label{decom:overview}
\end{figure}
As shown in Figure \ref{decom:overview}, our method can be separated into three steps: layer decomposition, brush stroke extraction, bas-relief generation. \\
\textbf{Layer decomposition:}\\ A given painting is firstly decomposed into a set of layers in terms of paint colors. In our decomposition, each layer represents a single-color coat of paint applied with varying opacity.\\ 
\textbf{Brush stroke extraction:} \\ Secondly, each resulting layer is further segmented into multiple regions which represent brush strokes respectively.As shown in Figure \ref{decom:overview}, we use different colors on a black image to show the results of our brush stroke extraction, and each color(except black) represents the region of a single extracted brush stroke. \\
\textbf{Bas-relief generation:} \\ Thirdly, the depth map of each brush stroke is generated individually by the shape from shading techniques. After that, the desired bas-relief is generated by merging all the depth maps of brush strokes together. \\ \\ 

The key point is to extract brush strokes from input paintings. Overlapped strokes make colors blend. To deal with it, layer decomposition is firstly employed here, which decomposes the painting into a set of single colored and translucent layers.In this research, we assume a brush stroke only utilizes a single paint color. Therefore, layer decomposition helps classify brush strokes separately into different layers based on their colors so that every layer contains the brush strokes which are better separated.\\
However, wrong layer decomposition may cut one stroke into two or more layers. It is observed that most brush strokes in Chinese paintings the color and opacity change very little in the direction of the stroke \cite{xu2006animating}. So, we introduce the edge tangent flow (ETF) field and the coherent line \cite{kang2007coherent} to enhance such features in paintings, which is in favor of preserving the wholeness of the strokes in every layer and effectively avoid wrong layer decomposition.
\newline
The coherent line is further involved in the maximally stable extremal regions(MSER) algorithm \cite{donoser2006efficient} again for better extraction within one layer.\\
Furthermore, to generate the depth maps of the strokes individually, we perform shape from shading(SFS) on the opacity of the paintings instead of the intensity here, since the opacity has a bigger range than the intensity (as can be seen in Figure \ref{histo} for details). SFS techniques may generate details of surfaces in terms of image texture. It is desirable to transfer the features of the paintings, e.g. the density of colors, to the surface of the brush stroke models.
\newline
The depth maps of all the strokes are then merged together to form the desired bas-relief. We hope to point out that the employed orthographic SFS technique tends to encourage interior growth within a closed region, which may result in the growth of a background region surrounded by strokes. In general, the background plane should be unchanged. Thus, generating strokes individually not only benefits the composition of bas-reliefs but also avoid this technique issue.
Moreover, the stroke order and shape can be further edited,since the user may want to reform the bas-relief models for secondary creation.
\newpage 