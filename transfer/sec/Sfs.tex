\subsection{Computer vision based methods}

Shape recovery is a classic problem in computer vision.The goal is to derive a 3D scene description from one or more 2D images. The computation process normally involved with several concepts : depth $Z(x,y)$, surface normal $n_x,n_y,n_z$ ,and surface gradient $p,q$. The depth can either be considered as distance from viewpoint to query surface or the height from surface to default $x-y$ plane. The normal is perpendicular to the surface gradient, namely $\left( n_x,n_y,n_z\right) * \left( p, q ,1\right)^T =0  $ , the surface gradient is the changing rate of depth in $x$ and $y$ direction. 

Shape from shading (SFS) is able to recover the shape of an object from a given single image, assuming illumination and reflectance are known (or assuming reflectance is uniform across the entire image). Many methods have been developed, which may be categorized into four PDEs models\cite{prados2003perspective}, (1) orthographic SFS with a far light source \cite{lions1993shape}; (2) perspective SFS with a far light source\cite{prados2004unifying}; (3) perspective SFS with a point light source at the optical center\cite{prados2003perspective}; (4) a generic Hamiltonian. However, SFS is an ill-posed problem. The notable difficulty in SFS is the bas-relief ambiguity\cite{belhumeur1999bas}, that is, the absolute orientation and scaling of a surface are ambiguous given only shading information. To amend it, many SFS algorithms impose priors on shape, depth cues produced by a stereo system, or assume that the light source, the surface reflectance, and the camera are known\cite{zhang1999shape}; \cite{alldrin2007resolving}; \cite{johnson2011shape}; \cite{barron2012color}; \cite{han2013high}.
However,\cite{sykora2014ink} argued that “the knowledge of accurate depth values is not necessary to produce believable advanced illumination effects”. They further presented an approach for generating global illumination renderings of hand-drawn characters through a bas-reflief type mesh. The key point is to create such a mesh from a hand-drawn image using only the contours and the layering relationships of the components. Unfortunately, the resulting mesh looks inflated since it is generated only by Dirichlet and Neumann boundary constraints.


To recover the shape of scene from shading of single image,it's important to exploit how image formed from lighting and shading.One simplest model is Lambertian shading model, in which the intensity of image are purely depends on the normal of surface and light direction. Given the intensity of each pixel of image, SFS aims to recover the light direction and surface shape. Meanwhile, in real world, situation is more complex than the Larbertian shading model,even with known light direction, and the gray level can be described as a function of surface shape and light source direction, the problem is still not simple. 

If the surface shape is described in terms of the surface normal, we have a linear equation with three unknowns, and if the surface shape is described in terms of the surface gradient, we have a nonlinear equation with two unknowns. Therefore, finding a unique solution to SFS is difficult; it requires additional constraints.

To solve the SFS problem, there are four common ways: minimization approaches, propagation approaches, local approaches, and linear approaches \cite{zhang1999shape}. Minimization approaches, namely, focus on how to form a energy function and obtain the solution by minimization. Propagation starts certain points of the surface. Local approaches based on assumption of surface type. Linear approaches compute a linearization from reflection model. 

\subsubsection{Minimization Approaches}
Ikeuchi and Horn \cite{ikeuchi1981numerical} proposed the method to recover the surface gradient. Since each surface point has two unknowns for the surface gradient and each pixel in the image provides one gray value, we have an underdetermined system. To solve this problem,  the brightness constraint and the smoothness constraint were introduced . The brightness constraint aims to produce the shape that generate the same brightness of input image, and the smoothness constraint constrains the smoothness of reconstructed shape. By minimizing these two constraints,the shape was computed. To ensure a correct convergence, the shape at the occluding boundary was given for the initialization.Because on the boundary of objects the gradient is infinite, stereographic projection was also applied to transform the the error function to a different space.Using these two constraints, Brooks and Horn \cite{brooks1985shape} also use minimization method,in terms of the normal of surface. The integrability was enforced in \cite{frankot1988method} to generate integrable surface.

Surface slope estimates from the iterative scheme were expressed in terms of a linear combination of a finite set of orthogonal Fourier basis functions. The enforcement of integrability was done by projecting the nonintegrable surface slope estimates onto the nearest (in terms of distance) integrable surface slopes.
This projection was fulfilled by finding the closest set of coefficients which satisfy integrability in the linear combination. Compared with the earlier research, their method is more accurate and efficient. Szeliski \cite{szeliski1991fast} applied a hierarchical basis preconditioned conjugate gradient descent algorithm .To improve the stability of Brooks and Horn’s algorithm a heuristics also was applied to the variational approach by \cite{vega1993shading}.

Intensity gradient constraint\cite{zheng1991estimation} was introduced to constraint the difference of gradient between the reflect map and input image in both x and y direction , which was used to substitute the smooth constraint. 

While the above mentioned method are based on variational calculus.
\cite{lee1993shape} focusing on calculate depth based on  discrete formulation and conjugate gradient method. To achieve convergence, brightness constraint and smoothness constraint should be applied.  Lee and Kuo \cite{lee1993shape} proposed a method  does not need to initialization of depth,and they approximated surfaces by a union of triangular patches. 

Above mentioned approaches are based on producing a single smooth surface. Malik and Maydan \cite{malik1989recovering} focused on piecewise smooth surfaces. To minimize the energy function, line drawing and shading constraints were applied, and by such, they finally recover both surface normal and line labeling.

\subsubsection{Propagation Approaches}

Horn proposed the characteristic strip method \cite{horn1990height} which is based on propagation.In  characteristic strip method, if the at the beginning point of the characteristic strip  line , surface depth and orientation is already known, then along the line all the other points along the line can be computed. Characteristic strip method starts with singular points, pixels with maximum intensity, and construct their neighborhoods based on sphere approximation which results in some initial surface curves. The direction of the strips are determined by the local gradient , and along the characteristic strips the depth information would be propagated outwards. Since only by the singular points, the correspond curves are relatively sparse, to produce dense shape, new characteristic strips were interpolated from the initial strips. 

Hamilton-Jacobi-Bellman equations and viscosity solutions theories was applied in the work of Rouy and Tourin \cite{rouy1992viscosity} , and by such assumption, we can have an unique solution.
Some other researches proves that surface reconstruction can starts from singularity points instead of occluding boundaries \cite{oliensis1991shape},  and based on such idea, Shape from shading can be formulated as the optical control problem , and it can be solved by numerical methods \cite{oliensis1993global}. Based on  Dupuis and Oliensis's approach, a minimum downhill method shows a improvement on efficiency which could achieve convergence in 10 iteration \cite{dupuis1992direct} .  

Similar with these methods, starting with a close curve in the areas of singular points, Kimmel and Bruckstein \cite{kimmel1992shape} proposed a method  based on layers of equal height contours to reconstruct the surface.By applying differential geometry, fluid dynamics, and numerical analysis, their outcome enabled nonsmooth surface generation. 


\subsubsection{Local Approaches}
In the work of Pentland \cite{pentland1984local}, the first and second derivatives were applied for the surface reconstruction and base on the assumption of every local points is a spherical. Similarly, under this locally spherical assumption,  and by using a light surce coordinate system, slant and tilt of the surface can be computed \cite{lee1985improved} from the first derivative of the image. 

\subsubsection{Linear Approaches}
Some works mentioned above are based on linear approaches which aim to reconstruct the depth map from a linearization based on the reflectance. By Fourier transfer the linear function can have a closed from solution at each point of depth \cite{pentland1989shape} , and we can have such a linear approximation of the reflectance map from local gradient. 
Jacobi iterative scheme was also been used for recovering the surface at each pixel in the work of Ping-sing and Shah \cite{ping1994shape},and they used the discrete approximation of the gradient first, then employed the linear approximation of the reflectance function in terms of the depth directly.

\subsubsection{Interreflections}
While above mentioned researches simply recover the surface based on the reflect function,in the works of Nayaret et al. \cite{nayar1989surface} \cite{nayar1990shape} inter-reflection has also been put into consideration. Based on their observation, the shape reconstructed from shape-from-photometric-stereo algorithms has erroneous, in the area which effected by inter-reflection, the recovered shape is shallower than the real shape.  So,their method focused on this problem based on iteratively refinement. Similar method was proposed by Forsyth and Zisserman \cite{forsyth1989mutual}.

\subsubsection{Convergence}
Some researches focus on the convergence of the SFS function, based on the work of Oliensis \cite{oliensis1991shape} , the solution has uniqueness when the light source direction is equal to or symmetric to view source. If the first derivative of the surface is continuous, the characteristic strip could yield a unique solution. Commonly, the uniqueness of SFS problem is unknown. But in the work of Lee and Kuo \cite{lee1993shape}, with given depth of some points, and constrains of the smoothness, they can obtain a convergence from most cases based on their linearization of the reflection function. 