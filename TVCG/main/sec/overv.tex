
\section{Overview of proposed Research}
As shown in Figure 1, a given painting is firstly decomposed into a set of layers in terms of several specified palette color values (see the “decomposed layers” in Fig.1). Secondly, a new palette color value is determined based on the unclassified regions. Thirdly, the layers are recomputed accordingly in an iterative way (see the “refined layers” in Fig.1). After that to make the regions like separate brushstrokes, they are further merged or split.
The key point is to extract overlapped brushstrokes. Overlapped strokes make colors blend. To tackle it, layer decomposition is employed here, which decomposes the painting into a set of translucent layers. In brush paintings, each brushstroke only utilizes a single palette color. Layer decomposition helps classify brushstrokes separately into different layers based on the palette color so that every layer contains the strokes which are well separated.
However, wrong layer decomposition may cut one stroke into two or more layers. It is observed on multiple layers that typically brushstrokes of such paintings follow same patterns. For instance, a scan of Rosemaling painting employs many C and S brushstrokes, and the color and transparency change very little in the direction of the stroke. However, due to wrong palette color, some brushstrokes appear on multiple layers. We introduce the edge tangent flow (ETF) field and the coherent line [CITATION Kan07 \l 2057 ] to enhance such features in paintings, which is in favor of preserving the completeness of the strokes in every layer and effectively amend wrong layer decomposition.
Moreover, for the scanned physical paintings, it is impossible to exactly decompose paints into a limited number of layers. We therefore develop an iterative scheme to refine the layers.
The overlapped regions of multiple strokes usually result in the gaps to break the strokes within one or more layers. To tackle this challenge, inpainting technique is employed here.
The coherent line is further involved in the MSERs algorithm [CITATION Don06 \l 2057 ] again for extracting strokes, which both preserves stroke continuity and removes spurious edges within one layer.