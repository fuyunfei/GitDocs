
\section{Related work}
To the best of our knowledge, there is lack of study on extracting brushstrokes from the scanned physical paintings. We give a brief overview of the work related to the relevant topics, i.e. decomposing images into layers and stroke segmentation, which are employed in our implementation. In digital image editing, artists deposit color throughout the image via a set of strokes, which stay at the individual layers. However, scanned paintings and photographs have no such layers. Without layers, simple edits may become very challenging.[CITATION Ric14 \l 2057 ] presents an approach to produce editable vector graphics, in which the selected region is decomposed into a linear or radial gradient and the residual, background pixels. [ CITATION XuS06 \l 2057 ] aims at decomposing Chinese paintings into a collection of layered brushstrokes with an assumption that at most two strokes are overlapping and there is minimally varying transparency. [CITATION McC09 \l 2057 ],[CITATION McC12 \l 2057 ] present two generalized layer decomposition methods, which allow pixels to have independent layer orders and layers to partially overlap each other.[ CITATION Tan16 \l 2057 ] present a layer decomposition method based on RGB-space geometry. They assume that all possible image colors are convex combinations of the paint colors. Computing the convex hull of image colors and per-pixel layer opacities is converted into a convex optimization problem. Thus, their method can work well without prior knowledge of shape and overlap of strokes. Additionally, for oil paintings especially van Gogh’s artefacts, [Jia Li and Fabrizio] presented the individual supervised/unsupervised extraction schemes to extract brushstrokes for artwork authentication and artist identification purposes. Usually, there is no transparency in the brushstrokes of oil paintings, which is in favor of detecting the boundaries of brushstrokes but becomes hard to deal with occlusion and stroke order issues.
Currently, most research focuses on stroke segmentation of handwritten characters, such as pen strokes[ CITATION Her14 \l 2057 ]. Pen stroke edges are distinct and can be extracted entirely in handwritten characters. As a result, the basic idea is to describe a stroke by a set of geometric primitives, so that the hand-drawn primitives may be replaced by mathematically precise shapes to produce a neat final result. However, brushstrokes on a painting often contain individual colors and overlap each other. In fact, the strokes may show a low contrast to the others or the background. The edges of strokes are blurred.[ CITATION XuS06 \l 2057 ] extracts descriptions of the brushstrokes by using a brushstroke library. However, their approach requires a good amount of prior knowledge of shape and order of strokes.
The Most Stable Extremal Regions (MSERs) algorithm[ CITATION Mat04 \l 2057 ] was used for establishing correspondence in wide-baseline stereo.[CITATION Don06 \l 2057 ] introduced the data structure of the component tree in it and further developed it as an efficient segmentation approach, which prunes the component tree and selects only the regions with a stable shape within a range of level sets. The revised version has been widely used in stroke segmentation of handwritten characters [ CITATION Gom16 \l 2057 ].