
\section{Conclusions}
Extracting the strokes from a scanned physical painting is a hot and challenging task whether in fine art community or in computer vision and graphics field. We aim at brush paintings and deliver a novel the brushstroke extraction method. Since the extracted brushstrokes are independent of each other, the order and shape may be redefined by the users. It is suitable for recomposition in art designs.\\
\textbf{Limitations}  
In our implementation, we employ layer decomposition in order to split the overlapped brushstrokes into different layers, so that the strokes can be easily segmented within one layer. However, the main challenge is the overlapped strokes may appear on the same layer in some brush paintings. Rosemaling painting designs use subtle and vibrant colors to enhance the color contrast between adjacent brushstrokes. Moreover, for a layered look, transparency is applied and is variable with a big scope. Thus, layer decomposition benefits the overlapped brush strokes separated on Rosemaling paintings. However, Chinese brush paintings do not emphasize the use of vibrant colors, even the adjacent strokes may share the same color in some paintings. The boundary of the overlapped brushstrokes may be very blurred. This may result in stroke segmentation failure.
Additionally, for oil paintings, there is almost no transparency to be applied. As a result, the segmented strokes may be incomplete due to overlap.\\
\textbf{Future work} 
In the future, we will study a wider range of paintings and investigate the issues of layer decomposition and stroke segmentation. Different paintings have different stroke patterns. Understanding the rules will improve the success rate of a wider range of paintings. Proper automatic color separation in the overlap regions is not trivial which will also be studied in the future.