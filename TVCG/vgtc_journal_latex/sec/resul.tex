\section{ RESULTS AND ANALYSIS}
We use the published codes of the layer decomposition and MSERs, which are available on GitHub (at: https://github.com/CraGL/Decompose-Single-Image-Into-Layers; and https://github.com/idiap/mser), and performed the proposed approach on eight brush paintings, including Rosemaling, van Gogh oil painting and Chinese brush paintings. All the paintings are from the internet. Compared to Rosemaling, Chinese brush paintings do not emphasize the use of vibrant colors, even the adjacent strokes may share the same color in some paintings. The boundary of the overlapped brush strokes may be very blurred. Thus, we select some of the Chinese brush paintings for a test in terms of the boundary of strokes and the use of transparency. The oil paintings, such as van Gogh’s paintings, are usually composed of lots of small and similar brushstrokes. Too much small strokes bring about difficulties for animation purposes. All the tests were performed on a 6-core of 3.33 GHz Intel Core Xeon CPU with memory of 32 GB(RAM).\\
In our implementation, the parameters in Layer decomposition, ETF field, and MSERs are set the default values as in the original codes. Table 2 further shows the running time of the proposed approach. Compared to the performance in [ CITATION Tan16 \l 2057 ] and [ CITATION Nis08 \l 2057 ], there is no distinct difference. To demonstrate the extracted strokes, we fulfill three tests, including recoloring paintings, inserting objects and animating strokes. Our implementation is not multithreaded. All the resulting images/videos are available in the supplementary materials.
\subsection{Recoloring Paintings}
It is natural to recolor the specified strokes for recomposition. Once the brushstrokes are extracted, recoloring strokes with a new palette color is becoming as simple as linearly combining N images.\\
Figure 9b and 9c shows recoloring strokes on three paintings respectively, Rosemaling, van Gogh oil painting and Chinese bush painting. As the strokes have been extracted, it is easy to separately recolor one or more strokes with different colors. However, for van Gogh oil painting, it seems tricky, since the painting is composed of lots of brushstrokes (more than one hundred, see Figure 10). Thus, we only recolor one of the layer’s colors rather than that of some brushstrokes here.\\
Moreover, Figure 10 further illustrates the brushstrokes of the van Gogh oil painting, which justifies the efficiency of our brushstroke extraction method.
\subsection{Inserting Objects}
Figure 9d shows stroke manipulation through inserting objects for recomposition. One of image synthesis tasks is to change a specified region with a new object in a seamless and effortless manner. Here we are interested in inserting new objects into a painting while keeping the transparency of the painting. Note that the inserted objects are opaque and are inserted in between two brushstrokes here. The occluded regions of the objects are visible due to the transparency of the brushstrokes. Unlike the traditional image synthesis approaches, our implementation works on the strokes, which both guarantees seamless and keep the transparency of the painting.
\subsection{Animating Strokes}
Usually, the number of strokes in each painting is small and hence manageable. To illustrate brushstrokes, we perform animating one stroke and multiple strokes respectively in Figure 11. Users can select and move any control point of either the skeleton or the contour of the stroke to be animated.