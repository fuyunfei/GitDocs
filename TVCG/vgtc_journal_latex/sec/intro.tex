\firstsection{Introduction}

\maketitle

%% \section{Introduction} %for journal use above \firstsection{..} instead
A popular impression is that a painting is made up of different types of painting strokes and is rather like learning write. One brush may create different effects using the different strokes. Nowadays artists may select real or simulated (in the case of digital paintings) painting strokes for their paintings. An inverse issue is rising, how to extract brushstrokes from the scanned physical paintings. This issue is of great practical signification. For example, like background replacement in image and video editing, it requires accurate extraction of foreground objects  [Qifeng Chen, Ehsan Shahrian]. Moreover, it is desired for digital painting software, e.g. Adobe Photoshop, to simulate the act of painting in the real word, not only painting by using a virtual brush but also extracting a set of hypothetical brushstrokes from an image. For animation applications, it aims to animate a painting by decomposing an image of the painting into its hypothetical brushstroke constituents [Songhua Xu]. From a computational point of view, the final painting is a grid of unstructured color values. The main challenge is how to extract each stroke from the occluded or overlapped brushstrokes in the painting.
Image processing has been experimented as an effective mean for an in-depth visual analysis of high-level features related to brushstrokes of paintings. It is believed that certain patterns from images can be extracted in a computational scheme more thoroughly than through manual attempts [Jia Li]. For example, in digital image editing systems.(e.g. Photoshop), layers organize images. However, for a photograph, layers may never have existed. Research efforts to decompose the photograph into layers have emerged in the recent years[ CITATION Tan16 \l 2057 ].
Consider a kind of the scanned physical paintings, e.g. a brush painting, which is produced with individual strokes and each stroke may belong to the individual layers. It is possible to extract brushstrokes from the scanned images.
The aim of our research is to extract the brushstrokes from brush paintings and further animate them for animation application.
This however demands to conquer several challenges. First, each brushstroke covers a region on the canvas and they may overlap each other, some quite heavily in a painting. To make sure the information is retained, every stroke has to be faithfully extracted. Second, spatial occlusion has to be dealt with, since artists are used to depicting it through controlling the transparency of strokes as one of the art elements. Third, for animation purposes, strokes should be further editable allowing the artist to rearrange, tweaking and reshaping them.
The shape, color and opacity of a stroke vary due to the shape and firmness of the brush as well as the forces the artist imposes. Although these variables add the complexity to stroke extraction, stylized strokes often follow distinct patterns. For example, Rosemaling paintings, a typical example of brush painting popular in North Europe, is a traditional form of decorative folk art that originated in the rural valleys of Norway. The Rosemailing designs use C and S strokes, feature scroll, flowing lines, floral designs, and both subtle and vibrant colors[ CITATION Pat77 \l 2057 ]. The brush strokes may further be viewed as graphical objects which are meaningful with respect to the objects the painting portraits. Moreover, each stroke is clearly visible due to both subtle and vibrant colors. The similar properties may be found in some Chinese brush paints and oil paintings such as those of van Gogh.
To extract the strokes from a brush painting, it is essentially an inverse problem. The main challenge is that such inverse problems are usually ill-posed. Our contributions include,
(1) Layer decomposition. We propose a novel iterative scheme for layer decomposition. The distinct advantage is that the strokes are classified into as different layers as possible, whilst the iterative scheme converges quickly.
(2) Brushstroke extraction. We develop a novel method which may entirely construct every brushstroke based on the opacity of strokes. This enables recomposition in design by redefining the brushstrokes’ order and shapes, or even animating them.